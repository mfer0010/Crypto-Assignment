\chapter{Block Cipher Modes and Message Authenticity}

\section{Introduction}
\paragraph{ }In this task, a secure communication system will be modelled that makes use of an AES library for an e-Banking application for money transfers. The text-based protocol that's used by this e-Banking company has the following message format:

\par\centerline{\texttt{Accnt:}\textit{[10 bytes]}\texttt{Accnt:}\textit{[10 bytes]}\texttt{Descr:}\textit{[58 bytes]}\texttt{Amount:}\textit{[9 bytes]}}

\paragraph{ }The following libraries and functions were imported in order to run all the code presented in this chapter:

\begin{lstlisting}
from Crypto.Cipher import AES
from Crypto.Util.Padding import pad
from Crypto.Util.Padding import unpad
from Crypto.Random import get_random_bytes
from base64 import b16encode
from base64 import b16decode
import os
import json
import hmac
import hashlib
\end{lstlisting}

\paragraph{ }The required data was then initialized as follows:

\begin{lstlisting}
#Initialize Data
data = b"Accnt:0003336669Accnt:0123456789Descr:00000000000000
		00000000000000000000000000000000000000000000A
		mount:023456789"
key = get_random_bytes(16)
print(b16encode(key).decode('utf-8'))
\end{lstlisting}


\section{ECB Mode Implementation}
\paragraph{ }The first task was to implement the e-Banking system using AES in ECB mode and demonstrate the weakness of using this approach through an attack which swaps the accounts thus reversing the direction of the transfer.

\paragraph{ }The data was encrypted in ECB mode using the following code:

\begin{lstlisting}
ECB_cipher = AES.new(key, AES.MODE_ECB) #Create a new AES Cypher
 #encrypt the padded data
ciphertext_bytes = ECB_cipher.encrypt(pad(data, AES.block_size))
#Encode the cyphertext to 16 bit encoding to have a more clear output
#use the decode function to remove the b'...' in the output
ciphertext = b16encode(ciphertext_bytes).decode('utf-8')
print(ciphertext)
\end{lstlisting}
\textit{Output:} \texttt{4CDEB5515885FEC7390A48D4F9747BD53A309C3CA17838A5A1FD18A9A433737E9}\\
\texttt{D35FE123067946375B9B59FAD318E1CE9C1DC732606DAA7644898E4D577B90DE9C1DC732606}
\texttt{DAA7644898E4D577B90DE9C1DC732606DAA7644898E4D577B90DBD18D9394E2B939E4EA7D9E}
\texttt{8C14C0B04FAF69C2C9877EE7A8805287F5B34C232}

\paragraph{ }The ciphertext was then immediately decrypted to ensure it was encrypted successfully:
\begin{lstlisting}
decoded_ciphertext = b16decode(ciphertext) 
decrypted_text = unpad(ECB_cipher.decrypt(decoded_ciphertext),
				 AES.block_size)
print(decrypted_text)
\end{lstlisting}
\textit{Output:} b'Accnt:0003336669Accnt:0123456789Descr:00000000000000000000000000000000\\
00000000000000000000000000Amount:023456789'

\paragraph{ }The attack was then demonstrated by Oscar by simply swapping the desired bits of the ciphertext, and once Bob decodes the message he gets a valid transaction and is thus none the wiser that the accounts have been swapped.
\begin{lstlisting}
#Swap the account numbers to reverse the direction of the transfer
modified_ciphertext = ''.join([ciphertext[32:64],
	ciphertext[0:32], ciphertext[64:]])
#decrypt message:
modified_result = unpad(ECB_cipher.decrypt(b16decode
	(modified_ciphertext)), AES.block_size)
print(modified_result.decode('utf-8'))
\end{lstlisting}  
\textit{Output:} Accnt:0123456789Accnt:0003336669Descr:00000000000000000000000000000000000\\
00000000000000000000000Amount:023456789

\section{CBC Mode Implementation}
\paragraph{ }The encryption was then implemented using CBC mode and another attack was demonstrated by modifying the amount since the implementation does not verify for valid format field labels.

\paragraph{ }The data was encrypted in CBC mode using the following code:

\begin{lstlisting}
iv = os.urandom(16) #initializing vector
#Create new AEC Cipher in CBC mode
CBC_cipher = AES.new(key, AES.MODE_CBC,iv) 
#Encrypt padded data
ciphertext_bytes = CBC_cipher.encrypt(pad(data, AES.block_size)) 
#Encode the cyphertext to 16 bit encoding to have a more clear output
#use the decode function to remove the b'...' in the output
CBC_ciphertext = b16encode(ciphertext_bytes).decode('utf-8')
#Save as Json in order to simulate passing the whole thing over the network
json_data = json.dumps({'iv':b16encode(iv).decode('utf-8'),
	 'ciphertext':CBC_ciphertext})
\end{lstlisting}

\paragraph{ }The ciphertext was again immediately decrypted to ensure it was encrypted successfully:

\begin{lstlisting}
#Read the data from the json file
b16 = json.loads(json_data)
iv = b16decode(b16['iv'])
#Create a new Cipher for decryption using the same key and iv
CBC_cipher2 = AES.new(key, AES.MODE_CBC, iv)
#Decode, unpad and display the results
CBC_dec_ciphertext = b16decode(b16['ciphertext'])
CBC_decoded_plaintext = unpad(CBC_cipher2.decrypt(
	CBC_dec_ciphertext),AES.block_size)
print(CBC_decoded_plaintext)
\end{lstlisting}
\textit{Output:} b'Accnt:0003336669Accnt:0123456789Descr:00000000000000000000000000000000\\
00000000000000000000000000Amount:023456789'

\paragraph{ }An XOR function was then defined to be used in the attack:
\begin{lstlisting}
def xor(A, B):
	return hex(int(A, 16) ^ int(B, 16))[2:].upper() 
\end{lstlisting}

\paragraph{ }The attack as explained previously was then demonstrated as follows:
\begin{lstlisting}
#Attacker can also reda the data from the json file
Oscar_b16 = json.loads(json_data)
Oscar_ciphertext = Oscar_b16['ciphertext']
Oscar_iv = Oscar_b16['iv']
block_to_alter = Oscar_ciphertext[160:192] #block containing amount
byte_to_change = block_to_alter[14:16] #amount

#xor 1st bit of plaintext with the hex value you want in new plaintext
m = xor('0x30', '0x39') #0x30 is decimal 0, 0x39 is decimal 9

#perform bit flipping
new_byte = xor(hex(int(m, 16)), hex(int(byte_to_change, 16)))
while len(new_byte) < 2:
	new_byte = "".join(['0', new_byte])
attacked_ciphertext = ''.join([Oscar_ciphertext[0:174],
		 new_byte, Oscar_ciphertext[176:]]) 

result = json.dumps({'iv':Oscar_iv, 'ciphertext':attacked_ciphertext})
\end{lstlisting}

\paragraph{ }Finally, the attack was decoded by Bob and as can be seen in the output, the amount was successfully altered.
\begin{lstlisting}
#Create a new Cipher for decryption using the same key and iv
CBC_cipher_Oscar = AES.new(key, AES.MODE_CBC,b16decode(Oscar_iv))
#Decode, unpad and display the results
CBC_dec_ciphertext = b16decode(attacked_ciphertext)
CBC_decoded_plaintext = unpad(CBC_cipher_Oscar.decrypt(
		CBC_dec_ciphertext),AES.block_size)
print(CBC_decoded_plaintext)
\end{lstlisting} 
\textit{Output:}b'Accnt:0003336669Accnt:0123456789Descr:000000000000000000000000000000\\
000000000000\textbackslash xbbs\textbackslash x14\textbackslash x81\textbackslash x92xC\textbackslash x975\textbackslash xbfb\textbackslash xdf\textbackslash xaa\textbackslash \textbackslash \textbackslash xc8DAmount:923456789'

\section{Hardened Implementation}
\paragraph{ }The system was then hardened in order to provide transaction integrity. Message authentication codes were used to protect from content tampering and a challenge-response was st up to protect from replay attacks.

\paragraph{ }Initially, a function was set up to compare the MACs:
\begin{lstlisting}
def compare_mac(a, b):
	a = a[1:] 
	b = b[1:]
	different = 0

	for x, y in zip(a, b):
		different |= x ^ y
	return different == 0
\end{lstlisting}

\paragraph{ }The data was then encrypted in CBC mode but with added MAC using HMAC-SHA256 standard.
\begin{lstlisting}
hmac_key = get_random_bytes(16)
plaintext = pad(data, AES.block_size)
iv_bytes = get_random_bytes(AES.block_size)
HMAC_cipher = AES.new(key, AES.MODE_CBC, iv_bytes)
encrypted_data = HMAC_cipher.encrypt(plaintext)
iv = iv_bytes + encrypted_data #secret prefix mac
signature = hmac.new(hmac_key, iv, hashlib.sha256).digest()
\end{lstlisting}

\paragraph{ }Finally, the signature was verified and the data decrypted if it wasn't tampered with.
\begin{lstlisting}
new_hmac = hmac.new(hmac_key, iv, hashlib.sha256).digest()
if not compare_mac(new_hmac, signature):
print("Incorrect decryption")
cipher = AES.new(key, AES.MODE_CBC, iv_bytes)
dec_plaintext = cipher.decrypt(encrypted_data)
print(unpad(dec_plaintext, AES.block_size).decode('utf-8'))
\end{lstlisting}
\textit{Output}: Accnt:0003336669Accnt:0123456789Descr:00000000000000000000000000000000000\\00000000000000000000000Amount:023456789

\section{Demonstrating Failed Attacks}
\paragraph{ }Two failed attacks were then demonstrated to show the new system's security. The first attack attempts to tamper the data, while the second failed attack attempts to perform a replay attack.

\paragraph{ }The attack function attempts the data tampering similar to the first attack shown when the data was encrypted in CBC mode.
\begin{lstlisting}
#Like Task B's attack
def attack(data):
	b16 = json.loads(data)
	ciphertext = b16['ct']
	iv = b16['iv']
	block_to_alter = ciphertext[160:192] 
	byte_to_change = block_to_alter[14:16]

	m = xor('0x30', '0x39')

	#perform bit flipping
	new_byte = xor(hex(int(m, 16)), hex(int(byte_to_change, 16)))
	while len(new_byte) < 2:
		new_byte = "".join(['0', new_byte])
	new_ciphertext = ''.join([ciphertext[0:174],
			 new_byte, ciphertext[176:]]) 

	result = json.dumps({'iv':iv, 
			'ct':new_ciphertext})
	return result
\end{lstlisting}

\paragraph{ }The \textit{decrypt\_message} function performs the required checks before decrypting the data in order to ensure the data hasn't been tampered with, similar to the decryption step shown in the previous section.

\begin{lstlisting}
#Like Task C's Decryption
def decrypt_message(encrypted_data, iv_bytes, signature, shared_key,
			 hmac_key):
	iv = iv_bytes + encrypted_data
	new_hmac = hmac.new(hmac_key, iv, hashlib.sha256).digest()
	if not compare_mac(new_hmac, signature):
		print("Attack!")
	cypher = AES.new(shared_key, AES.MODE_CBC, iv_bytes)
	plaintext = cypher.decrypt(encrypted_data)
	return unpad(plaintext, AES.block_size)
\end{lstlisting}

\paragraph{ }The tamper attack was done as follows:
\begin{lstlisting}
iv = b16encode(iv_bytes).decode('utf-8')
ciphertext = b16encode(encrypted_data).decode('utf-8')
ct_json = json.dumps({'iv': iv, 'ct': ciphertext})
altered_ciphertext = attack(ct_json)
altered_ciphertext = b16decode(json.loads(altered_ciphertext)['ct'])
decrypt_message(altered_ciphertext, iv_bytes, signature, key, hmac_key)
\end{lstlisting}
\textit{Output:} Attack!

\paragraph{ }Finally, the replay attack was also demonstrated:
\begin{lstlisting}
altered_ciphertext = b"".join([encrypted_data[32:64],
	 encrypted_data[0:32], encrypted_data[64:]])
decrypt_message(altered_ciphertext, iv_bytes, signature,
	 key, hmac_key)
\end{lstlisting}
\textit{Output:} Attack!