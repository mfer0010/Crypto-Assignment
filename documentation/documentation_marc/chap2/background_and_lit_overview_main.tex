\chapter{Transparent Access Control on the Blockchain}

\section{Introduction}
\paragraph{ }In this task, a reference monitor was implemented where access request decisions took the form of Blockchain transactions. In particular, \textit{n} parties were provided access to a controlled resource through a shared secret \textit{s}. Fiat-Shamir's Zero Knowledge Proof was used in order not to disclose the shared secret \textit{s}.

\section{Implementing the Reference Monitor}
\paragraph{ }The reference monitor was implemented in Jupyter Notebook. The RSA modulus was generated using PyCryptodome's \textit{RSA.generate} function, where the number of bits was set to 2048 as this was deemed a sufficient length for 2017 and the public exponent \textit{e} was set to be 65537 as this is the FIPS standard. The generated key (which contains the RSA modulus $n=p.q$) was then saved to a file protected by a password in order to simulate that only Rene the reference monitor has access to it.

\begin{lstlisting}
%reset
from Crypto.PublicKey import RSA
import random
\end{lstlisting}

\begin{lstlisting}
#Setting up FS
#Generate n:
key = RSA.generate(2048,e=65537) 
#nbits = 2048 (Sufficient length for 2017), e -> FIPS Standard
#save the secure key to a file to only be accessed by Rene
passphrase = 'Str0ngPassw0rd!%'
f = open('mykey.pem','wb')
#Use a passphrase since
f.write(key.exportKey('PEM',passphrase=passphrase))  
f.close()
n = key.n #assume n is public knowledge

#Rene will then generate s, which lies in integer ring (Z_n)
s = random.randint(1,key.n+1)
#Assume this is sent to Alice and Bob over a secure channel

#Delete Secrets:
del key
del passphrase 
#Authorised parties will then be given mykey.pem and 
#with knowledge of the passphrase will have access 
#to the secure key
\end{lstlisting}

\paragraph{ } Once the RSA key was set up, Rene then set \textit{v} by using the equation $v=s^2$ mod $n$.

\begin{lstlisting}[language=Python]
#Rene:
#get the key
f = open('mykey.pem','r')
Rene_key = RSA.importKey(f.read(),passphrase='Str0ngPassw0rd!%')
f.close()
#set v
v = (s**2)%Rene_key.n
\end{lstlisting}

\paragraph{ }The interactive protocol was then implemented setting $t=100$, thus having a successful attack probability of $2^{-100}$.

\begin{lstlisting}[language=Python]
t = 100
for i in range(t):
	#Interactive Protocol
	#Alice
	#pick random r
	r = random.randint(1,n+1)
	x = (r**2)%n #compute x
	#x is now sent to Rene

	#Rene
	e = random.randint(0,1)
	#e is sent to Alice

	#Alice
	y = (r*(s**e)) % n
	#y is sent to Rene
	
	#Rene:
	if (y**2 % n != (x*(v**e))%n):
		print('Attack!')
		break
\end{lstlisting}

\section{Attack 1}

\paragraph{ }The implementation was then weakened in a way so that Alice always commits to the same $r \in Z_n$. An attack that discloses $s$ which hinges on the fact that whenever $e=0$ Alice sends $y=r$ \textit{mod} $n$ as a response was then shown. Given the fact that $r$ wasn't changing, Oscar could simply extract $r$ by intercepting $y$ when $e=0$. Once Oscar has $r$ then, when $e=1$, Oscar intercepts $y$ and can extract $s$ by using the equation $s=y.r^{-1}$ $mod$ $n$. The modified code to the previous implementation can be seen below.

\begin{lstlisting}[language=Python]
from sympy import mod_inverse

gotR = 0

t = 100
#Alice will pick a random r and keep it fixed:
r = random.randint(1,n+1)
for i in range(t):
	#Interactive Protocol
	#Alice:
	x = (r**2)%n #compute x
	#x is now sent to Rene

	#Rene
	e = random.randint(0,1)
	#e is sent to Alice on open channel, Oscar can intercept this
	
	#Alice
	y = (r*(s**e)) % n
	#y is sent to Rene on an open channel, Oscar can intercept this
	
	#Oscar's Attack:
	if e==0 and gotR == 0:
		oscar_r = y
		gotR = 1
	if e==1 and gotR == 1:
		oscar_s=(y*mod_inverse(r,n)) % n
		print(s==oscar_s)
		break

	#Rene:
	if (y**2 % n != (x*(v**e))%n):
		print('Attack!')
		break
\end{lstlisting}
\textit{Output:} True

\section{Attack 2}
\paragraph{ }The original implementation was then weakened in a way that Rene would always challenge with $e=1$. In this case, its possible to spoof Alice without knowing \textit{s}